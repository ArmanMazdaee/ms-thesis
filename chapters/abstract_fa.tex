\chapter*{چکیده}
هدف مسئله آوانویسی خودکار موسیقی، استخراج فرم نوشتاری موسیقی از روی موج صوت ضبط
شده از اجرای یک قطعه می‌باشد. به دلیل چالش‌هایی مانند گستردگی فضای خروجی، تفاوت
میان دایره‌زنگی سازهای مختلف و تداخل هارومونیک‌های صوتی این مسئله یکی از
سخت‌ترین مسائل حوزه بازیابی اطلاعات موسیقیایی می‌باشد.

با توجه به این که عمده داده موسیقیایی موجود به شکل موج صوت ذخیره شده است، داشتن
سیستمی که با قابلیت اطمینان بالا بتواند این اطلاعات را به فرم نوشتاری تبدیل کند
اهمیت بسیار بالایی دارد. همچنین حل بسیاری از مسائل حوزه بازیابی اطلاعات
موسیقیایی بر روی فرم نوشتاری موسیقی بسیار ساده‌تر است. درنتیجه دست‌یابی به یک
سیستم آوانویسی خودکار موسیقی که بتواند عملکرد خوبی داشته باشد، منجر به حل شدن
بسیاری از مسائله دیگر این حوزه می‌شود.

در بازشناسی خودکار گفتار استفاده از یک مدل زبانی در کنار مدل آوایی می‌تواند
عملکرد نهایی سیستم را به شدت افزایش دهد. موسیقی نیز مانند زبان طبیعی، ساختاری
کاملا قائده‌مند است و محققان بر این باور هستند که متخصصان موسیقی نه تنها با
استفاده از اطلاعات صوتی دریافتی بلکه بیشتر با بهره‌گیری از دانش موسیقیایی خود
فرآیند آوانویسی را انجام می‌دهد. با این وجود سیستم‌های موجود برای آوانویسی
خودکار موسیقی تنها بر اطلاعات موجود در سیگنال اکتفا می‌کنند و از مدلی جدا
برای تصحیح این تشخیص‌ها بر اساس ساختارهای موسیقیایی بهره نمی‌برند.

دلیل این عدم استفاده از یک مدل موسیقیایی را می‌تواند چالش‌های ذاتی این مسئله
دانست. به علت مشکلاتی همچنون طولانی بودن دنباله‌های ورودی و خروجی روش‌های موجود
برای استفاده از یک مدل موسیقیایی در کنار یک مدل آوایی عملا قابل استفاده
نمی‌باشند. در این پایان‌نامه تلاش شده است که یک روش نوین برای استفاده از مدل
موسیقیایی در کنار مدل آوایی، برای انجام آوانویسی خودکار موسیقی، بررسی شود و
عملکرد این روش با مدل آوایی تنها بررسی شود.