\begin{latin}
\chapter*{Abstract}
The goal of automatic music transcription is to extract a transcription of music
from the sound wave. Considering the challenges of this problem such as span of
the output space, the difference in the timber of the multiple instruments,
inteference of various harmonies, automatic music transcription is considered
one of the hardest problems in music information retrieval. Since most of
available musical data is stored as sound wave, having a reliable system which
could do automatic music transcription has a critical importance. Also this
would make solving most of the problems associated with music retrieval
information field much easier. Hence, creating an automatic music transcription
system with acceptable performance would lead to solving of multiple problems in
the field. In automatic speak recognition, usage of a language model along with
an acoustic model would enhace the final result significantly. Since music, the
same as natural language, is a regulated structure, sceintists believe musicians
transcribe not only using the sound waves, but also relying heavily on their own
musical understanding. But the available automatic music transcription systems
only use the info available in the signals and usage of a music model to correct
the system of outputs is not a common practice. The main reason of this
limitation is the inherit challenges of this problem. Challenges such as the
length of input and output sequences make the usage of a music model in addition
to an acoustic model not practical. In this thesis, a new music transcription
technique using a music model along with acoustic model is explored. 
\end{latin}