\chapter{آزمایشات انجام شده}
\section{مجموعه داده}
MAPS یک \gls{dataset} از فایل‌های \gls{MIDI} و فایل‌های صوتی هم‌تراز آن‌ها است
که برای مسائل \gls{atm} و \gls{mpe} طراحی و آماده‌سازی شده است
\cite{emiya2009multipitch}. فایل‌های صوتی در شرایط محیطی مختلف ضبط شده‌اند و از
استاندارد CD تبعیت می‌کنند. به این معنی که هر نمونه ۱۶ بیت دقت دارد و نمونه
برداری با دقت ۴۴ کیلوهرتز انجام شده است. این \gls{dataset} شامل حدود ۶۵ ساعت
فایل صوتی است و حجم کل \gls{dataset} در حدود ۴۰ گیگابایت است. MAPS تحت مجوز
\rl{Creative Commons} به صورت عمومی منتشر شده است.

برای تولید MAPS ابتدا مجموعه‌ای عظیم از فایل‌ها \gls{MIDI} جمع‌آوری شده است. سپس
از فایل‌های \gls{MIDI} جمع‌آوری شده استفاده شده تا فایل صوتی متناظر هر کدام
تولید شود. برای تولید فایل‌های صوتی کاملا هم‌تراز با فایل‌های \gls{MIDI} از دو
روش استفاده شده است.

عمده فایل‌ها توسط نرم‌افزار ایجاد شده‌اند و از هیچ ساز فیزیکی برای تولید آن‌ها
استفاده نشده است. برای این مجموعه از فایل‌ها ابتدا فایل‌های \gls{MIDI} پشت هم
قرار داده شده‌اند تا فایل‌های \gls{MIDI} کمتر ولی با طولی بیشتر به دست آید. سپس
خروجی صوتی هر فایل توسط نرم‌افزار \lr{Steinberg’s Cubase SX} شبیه‌سازی شده است.
برای ایجاد تنوع در فایل‌های صوتی، نرم‌افزار جهت انجام شبیه‌سازی برای فایل‌های
مختلف از نمونه اجراهای پیانوهای مختلف و در شرایط محیطی متفاوت استفاده کرده است.
در نهایت فایل‌های صوتی ایجاد شده با توجه به فایل‌های \gls{MIDI} ابتدایی قطعه
قطعه شده‌اند تا اجرای متناظر هر فایل به دست آید. دلیل استفاده از فایل‌های
طولانی‌تر بجای استفاده از فایل‌های اصلی این بوده است که نرم‌افزار استفاده شده به
صورت اتوماتیک قابل کنترل نیست و برای هر ایجاد هر فایل نیز به دخالت انسانی است.

برای بخش دیگر از فایل‌های صوتی از یک پیانو Disklavier استفاده شده است. این
خانواده از پیانوها می‌توانند از طریق ارتباط \gls{MIDI} فشرده شدن هر کلاویه را با
قدرت‌های مختلف شبیه‌سازی کنند. در نتیجه مانند این است که یک نوازنده با قدرتی
فراانسانی هر کلاویه را در زمان درست و با \gls{velocity} مشخص شده فشار دهد و
دقیقا در زمان خواسته شده آن‌را رها کند. همچنین برای ایجاد تنوع در فایل‌های صوتی
ضبط شده در گروهی از فایل‌ها میکروفن در فاصله نیم‌متری ساز قرار داده شده است و در
گروه دیگر میکروفن در فاصله‌ای مابین ۳ تا ۴ متری ساز قرار دارد.

نکته قابل توجه این است که فایل‌های \gls{MIDI} که برای تولید صوت در روش‌های مختلف
استفاده شده است ممکن است اشتراک داشته باشند. در نتیجه از یک قطعه ممکن است چندین
اجرای مختلف موجود باشد.

در این \gls{dataset} چهار دسته فایل \gls{MIDI} و اجراهای متناظر آن‌ها وجود دارد:
\begin{enumerate}
    \item فقط شامل اجرای مجرد نت‌های مختلف و اجراهای کوتاه
    \gls{monophonic} است. از این بخش از \gls{dataset} می‌تواند برای روش‌هایی
    مانند \gls{nmf} که نیاز به اجرای جدای نت‌ها دارند استفاده کرد.

    \item  شامل اجرای \glspl{chord} می‌شود که نت‌ها با هم هیچ ارتباط
    موسیقیایی ندارند و به صورت تصادفی در کنار هم قرار گرفته‌اند. این بخش از
    \gls{dataset} برای آزمایش عملکرد مدل‌های \gls{mpe} بدون استفاده‌ای از دانش
    موسیقی مناسب است.

    \item شامل اجرا \glspl{chord} رایج در موسیقی‌های غربی، مانند \glspl{chord}
    جز و یا \glspl{chord} کلاسیک، است. در نتیجه این بخش از \gls{dataset} برای
    سیستم‌هایی که از دانش موسیقیایی برای افزایش  دقت استفاده می‌کنند مناسب است.

    \item شامل ۲۳۸ قطعه موسیقی کلاسیک هست که در زمان انتشار \gls{dataset} تحت
    مجوز \rl{Creative Commons} منتشر شده بودند. تمام این فایل‌ها به صورت دستی
    پردازش شده‌اند تا هر نت کشش و \gls{velocity} مناسب را داشته باشد. این بخش از
    \gls{dataset} برای سیستم‌های \gls{atm} مناسب است.
\end{enumerate}

در این پایان‌نامه فقط از فایل‌های صوتی ضبط شده از اجرای Disklavier برای ارزیابی
استفاده شده است. دلیل این انتخاب این است که در دنیای واقعی هدف آوانویسی اجراهای
یک ساز واقعی است و اجراهای مصنوعی که توسط نرم‌افزار تولید شده‌اند اکثر جزییات یک
ساز واقعی را ندارند. همچنین از هیچ قطعه‌ای که در بخش ارزیابی وجود دارد، برای
آموزش هم استفاده نشده است تا سیستم فقط بر روی قطعات کاملا جدید ارزیابی شود. در
نهایت تنها از قطعات کامل در MAPS استفاده شده است.