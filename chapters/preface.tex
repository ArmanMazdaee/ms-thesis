\chapter*{پیشگفتار}
هدف مسئله آوانویسی خودکار موسیقی، استخراج فرم نوشتاری موسیقی از روی موج صوت ضبط
شده از اجرای یک قطعه می‌باشد. به دلیل چالش‌هایی مانند گستردگی فضای خروجی، تفاوت
میان دایره‌زنگی سازهای مختلف و تداخل هارومونیک‌های صوتی این مسئله یکی از
سخت‌ترین مسائل حوزه بازیابی اطلاعات موسیقیایی می‌باشد.

با توجه به این که عمده داده موسیقیایی موجود به شکل موج صوت ذخیره شده است، داشتن
سیستمی که با قابلیت اطمینان بالا بتواند این اطلاعات را به فرم نوشتاری تبدیل کند
اهمیت بسیار بالایی دارد. همچنین حل بسیاری از مسائل حوزه بازیابی اطلاعات
موسیقیایی بر روی فرم نوشتاری موسیقی بسیار ساده‌تر است. درنتیجه دست‌یابی به یک
سیستم آوانویسی خودکار موسیقی که بتواند عملکرد خوبی داشته باشد، منجر به حل شدن
بسیاری از مسائله دیگر این حوزه می‌شود.

هرچند که موسیقی از ساختاری بسیار قاعده‌مند تبعیت می‌کند و محققان بر این باور
هستند که این آوانویسی توسط انسان نه تنها بر اساس اطلاعات صوتی دریافتی بلکه بیشتر
بر اساس دانش از ساختارهای موسیقیایی انجام می‌شود، ولی متاسفانه هنوز شیوه مناسبی
برای استفاده از این قواعد در یک سیستم آوانویسی خودکار موسیقی ابداع نشده است.

در فصل اول این پایان‌نامه ابتدا به تعریف مفاهیم اولیه موسیقی پرداخته می‌شود و
سپس مروری کلی بر روی حوزه بازیابی اطلاعات موسیقیایی انجام می‌شود. همچنین در
ادامه به صورت دقیق‌تر مسئله آوانویسی خودکار موسیقی بررسی می‌شود. در فصل دوم
مروری بر مهم‌ترین روش‌ها و تحقیقات انجام شده بر روی این مسئله انجام می‌شود. فصل
سوم بر مفاهیم یادگیری ژرف تمرکز دارد. در فصل چهارم آزمایشات انجام شده برای
ارتباط یک مدل موسیقیایی به یک مدل آوایی جهت حل مسئله آوانویسی خودکار موسیقی
بررسی شده است. در نهایت در فصل آخر این پایان‌نامه نتیجه‌گیری از آزمایشات انجام
شده و پیشنهاداتی برای تحقیقات آینده مطرح شده است.