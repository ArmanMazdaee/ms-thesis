\chapter{بررسی کارهای مرتبط}
در طی چهار دهه گذشته روش‌های گوناگونی برای مسئله \gls{atm} پیشنهاد داده شده است.
با وجود این که هدف نهایی هر سیستم \gls{ATM} استخراج حالت نوشتاری موسیقی از روی
سیگنال‌های صوتی هست ولی اکثر راه‌حل‌های موجود تلاش می‌کند با رسید به یک هدف
میانی مسئله را حل کنند. با توجه به ماهیت این هدف‌های میانی و ساختار آن‌ها
می‌توان این روش‌ها را به چهار دسته کلی تقسیم کرد:
\begin{itemize}
    \item در سطح فرم
    \item در سطح نت
    \item در سطح جریان
    \item در سطح نماد
\end{itemize}

آوانویسی در سطح فرم یا \gls{mpe}، تخمین تعداد و \gls{pitch} نت‌های موجود در هر
یک فرم زمانی است. این فرم‌ها معمولا طولی در حدود چند میلی ثانیه دارند. هرچند این
تخمین معمولا برای هر فرم به صورت جدا انجام می‌شود ولی اطلاعات زمینه‌ای معمولا در
یک مرحله‌ پس‌پردازیش برای روی نتیجه به دست آمده اعمال می‌شوند تا دقت را افزایش
دهند. روش‌هایی که در سطح فرم تلاش می‌کنند که آوانسی را انجام دهند از مفهوم نت
استفاده‌ای نمی‌کنند و معمولا با هیچ فهموم موسیقیایی درگیر نمی‌شود.

حجم زیادی از روش‌های موجود در سطح فرم کار می‌کنند. روش‌هایی مانند پردازش سیگنال
به صورت سنتی \cite{emiya2009multipitch,su2015combining}، مدل‌های برپایه احتمال
\cite{duan2010multiple}، روش‌های برپایه بیز \cite{peeling2009generative}، \gls{nmf}
\cite{smaragdis2003non,vincent2009adaptive,benetos2013automatic,fuentes2013harmonic}
و روش‌های برپایه \gls{nn} \cite{sigtia2016end,kelz2016potential}. تمام این
روش‌ها نقاط قوت و ضعف خود را دارند و هنوز روشی به عنوان روش واحد انتخاب نشده
است. برای مثال روش‌های بر پایه پردازش سیگنال ساده‌تر و سریع‌تر هستند و قابلیت
تعمیم بالاتری به سازهای مختلف دارند. در حالی که روش‌های بر پایه \gls{dl} بر روی
یک ساز خاص، مثلا پیانو، به نتایج بهتری دست یافته‌اند. روش‌های بیضی می‌توانند
مدل‌سازی کاملی از فرآیند تولید صوت ارائه دهند، در حالی که به شدت پیچیده‌تر و
کندتر هستند.

آوانویسی در سطح نت، یک مرحله از \gls{MPE} سطح بالاتر هست از این جهت که این
روش‌ها فقط وجود یا عدم وجود نت در یک فرم را بررسی نمی‌کنند، بلکه فرم‌ها را به هم
وصل کرده و نت‌ها را در طول زمان بررسی می‌کنند. در \gls{ATM} معمولا هر نت را با
سه ویژگی \gls{pitch}، onset و offset مشخص می‌کنند \cite{klapuri2007signal}. با
توجه مبهم بودن offset در نت، در بعضی از سیستم‌ها، معمولا در نظر گرفته نمی‌شود.
در نتیجه مدل فقط \gls{pitch} و onset را پیشبینی می‌کنند.

حجم زیادی از روش‌های در سطح نت معمولا پس‌پردازیش‌هایی بر روی خروجی یک سیستم
\gls{MPE} هستند. به عنوان روش‌های استفاده شده می‌توان از \gls{hmm}
\cite{nam2011classification} و \gls{nn} \cite{boulanger2012modeling} نام برد. در
پس‌پردازیش‌های انجام شده معمولا هر نمونه به صورت جدا بررسی می‌شود و رابطه بین
نت‌ها همزمان در نظر گرفته نمی‌شود که باعث تشخیص اضافه یا کمتر نت‌هایی می‌شود که
هارمونیک مشترک با نت‌های درست دارند. از این جهت روش‌هایی بر پایه مدل موسیقی
پیشنهاد شده است تا در ارتباط بین نت‌ها نیز در نظر گرفته شود
\cite{boulanger2012modeling, sigtia2016end}. بخش دیگری از روش‌ها نت‌ها را مستقیم
از روی سیگنال صوت استخراج می‌کنند. برخی دیگر ابتدا onset هر نت را پیدا می‌کنند و
سپس \gls{pitch} را بین آن‌ها تشخیص می‌دهند \cite{marolt2004connectionist}. برخی
دیگر حتی کل اطلاعات را با هم استخراج می‌کنند
\cite{cogliati2016context,ewert2016piano,hawthorne2017onsets}.

آوانویسی در سطح جریان یا \gls{mps} هدفش گروه بندی نت‌ها تخمین زده شده در
مجموعه‌ای جریان هست که هر جریان معمولا متناظر با یک ساز هست. این گروه از روش‌ها
ارتباط نزدیکی با \gls{iss} دارد. یکی از مزایای \gls{MPS} نسب به روش‌های قبلی
بررسی و تاثیر دادن \gls{timber} است. کارهای انجام شده در این سطح بسیار محدود
هستند.

تمام سه روش توضیح داده شده خروجی اصطلاحا پارامتری دارند. علت این نامگذاری این
هست که این آوانویسی انجام شده، یکی از پارامترهایش سیگنال صوتی ورودی هست. این
آوانویسی‌ها با وجود این که تا حدی از مفاهیم مسیقیایی استفاده می‌کنند ولی خروجی
آن‌ها هنوز اختلاف زیادی با سطح مجردسازی انجام شده در \gls{sheet music} دارد.
مهمترین این اختلاف‌ها زمان هست که در هر سه روش در واحد ثانیه اندازه گرفته
می‌شود. در حالی که در موسیقی زمان در واحد ضرب اندازه گرفته می‌شود. همچنین
\gls{pitch} در فراکنس هست در حالی که در موسیقی معمولا هر نت نامی مانند دو مینور
دارد و باتوجه به گام قطعه تعریف می‌شود. در نهایت مفاهیمی مانند میزان، ضرب آهنگ،
تمپو، گام و هارمونی کاملا حذف شده‌اند.

هدف آوانویسی در سطح نماد این هست که خروجی سیستم، نمادگذاری قابل لمس برای انسان
باشد که تمام اطلاعات موسیقیایی لازم را در بردارد. خروجی مانند \gls{sheet music}.
آوانویسی در این سطح نیاز به درک عمیق مفاهیم موسیقی مانند هارمونیک و ریتم دارد.
ساختارهای هارمونیک مانند گام و آکوردها باعث تغییر شکل نمایش \gls{pitch} می‌شوند.
ساختارهای ریتمیک مثل ضرب و میزان باعث می‌شود که طول نت‌ها از ثانیه مستقل شود.

چندین مطالعه تلاش کرده‌اند که ساختارهای موسیقیایی رو از روی سیگنال صوتی یا
\gls{MIDI} به دست آید انجام شده است. با این وجود مطالاعات خیلی کمی برای روی یک
سیستم آوانویسی کامل انجام شده است که از این مفاهیم استفاده کند.

با این وجود که روش‌های بسیار متفاوتی برای \gls{atm} وجود دارد ولی در دهه گذشته
بهترین دقت به دست آمده از دو خانواده از روش‌ها بوده است:
\begin{itemize}
    \item \gls{nmf}
    \item \gls{nn}
\end{itemize}
هر دو خانواده این مسائل برای مسائل بسیار متفاوتی مانند پردازش گفتار، پردازش
تصویر و سیستم‌های پیشنهاددهنده استفاده شده‌اند و نتایج بسیار رضایت بخشی به دست
آورده‌اند. در ادامه استفاده هر دو روش برای سیستم‌های \gls{ATM} بررسی می‌شود.