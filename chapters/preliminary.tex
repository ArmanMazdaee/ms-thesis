\chapter{تعریف مسئله و مفاهیم مقدماتی}
\section{بازیابی اطلاعات موسیقیایی}

پیشرفت‌های تکنولوژی مانند شبکه، دیسک فشرده، ذخیره‌سازی ابری و موارد مشابه باعث
ایجاد حجم عظیمی داده در شکل‌های مختلف شده است. با توجه به این حجم از داده، نیاز
به سیستم‌هایی خودکار جهت استخراج اطلاعات از این داده‌ها کاملا مشهود است. یک
خانواده روش‌های موجود برای طراحی این سیستم‌ها، روش‌های بازیابی اطلاعات مبتنی بر
محتوا%
\LTRfootnote{Content-based information retrieval}
 هستند. این روش‌ها به ما امکان می‌دهند که بر روی داده‌های چندرسانه‌ای به
جست‌وجو بپردازیم که از ضعف‌های سیستم‌های جست‌وجوی فعلی است. در چند دهه اخیر
CBIR
یکی از موضوعات پرطرفدار جهت مطالعه بوده است و ابزارهای مختلف زیادی نیز در این
حوزه توسعه یافته است که برای مثال می‌توان به جست‌وجو محتوا با عکس%
\LTRfootnote{Query-by-image content}
اشاره کرد. همچنین تکنولوژی بازیابی اطلاعات موسیقی%
\LTRfootnote{Music information retrieval}
کمک کرده است تا اطلاعات مورد نیاز از روی سیگنال‌های صوتی به دست آید. به صورت
کلی یک سیگنال صوتی، پدیده‌ای بسیار پیچیده است، زیرا شامل حجم زیادی اطلاعات
مرتبط با هنرمند، ژانر، احساس، ساز و غیره است. با توجه به تنوع بالا اطلاعات
موجود در این سیگنال‌ها، حیطه‌ی گسترده‌ای از مطالب در «بازیابی اطلاعات موسیقایی
مبتنی بر محتوا%
\LTRfootnote{Content-based music information retrieval}%
» جهت مطالعه و بررسی موجود می‌باشد.


برای بازیابی اطلاعات موسیقایی مبتنی بر محتوا کاربردهای زیادی مانند مجموعه
موسیقی شخصی‌سازی شده، توصیه موسیقی، دسته‌بندی موسیقی، محافظت حق چاپ و غیره
می‌توان متصور شد. برای توسعه این کاربردها نیاز است که فرا اطلاعات مورد نیاز از
فایل‌های موسیقی استخراج شود ولی از آن جا که متاسفانه این اطلاعات بر روی حجم
زیادی از فایل‌های موسیقی قرار داده نشده‌اند تنها راه موجود استخراج آن‌ها از روی
سیگنال‌های صوتی است. همچنین انتظار می‌رود بازیابی اطلاعات موسیقایی مبتنی بر
محتوا برای مسائل زیر راهکاری داشته باشد:
\begin{itemize}
\item تشخیص نواحی وکال در یک قطعه
\item تشخیص هنرمند، سبک و نوع یک قطعه
\item دسته بندی قطعات به دسته‌هایی مانند راک، جز، فولک، کلاسیک و غیره
\item نوشتن خودکار متن یک آهنگ
\item دسته بندی بخش‌های یک قطعه بر اساس بار احساسی آن بخش
\item تشخیص نوع سازهای استفاده‌شده در یک قطعه مانند زهی، کوبه‌ای و غیره
\item پیدا کردن قطعه‌های مرتبط به یک جستار به کمک معیارهای شباهت	
\end{itemize}

در ادامه به صورت خلاصه به معرفی بعضی از مهمترین مسائل این حوزه می‌پردازیم:

\subsection{قسمت‌بندی آواز و غیر آواز}
یک سیگنال صوتی از بخش‌های آواز، سازی، سکوت و یا ترکیب آواز و سازی تشکیل شده
است. تشخیص ساختاری یکی از مسائل جالب در این حوزه است. تشخیص نواحی سکوت یکی از
اولین قدم‌ها در هر سیستم پردازش گفتار و تشخیص صوت است. به صورت کلی طول بخش سکوت
در هر قطعه قابل چشم پوشی است زیرا بیش از ۹۹ درصد یک قطعه با صدای خواننده یا
صدای ساز پر شده است. بخش مشکل مسئله تشخیص آواز از صدای ساز است که در یک سیگنال
موسیقی در هم تنیده شده‌اند و بخش آوازی معمولا با یک موسیقی پس زمینه همراهی
می‌شود. این قسمت بندی پیش نیاز حل مسائل دیگری مانند تشخیص خواننده، تشخیص بار
احساسی، تشخیص ساز و رونویسی متن آواز است.

\subsection{تشخیص هنرمند}
هنرمند یکی از مهم‌ترین اجزای هر قطعه موسیقایی است. تشخیص خواننده، آهنگساز و
نوازنده از شاخه‌های این مسئله هست. در اکثر مواقع، سبک یکتای هر هنرمند، توسط
متخصصان قابل تمیز است. اگر با صدای یک خواننده آشنا باشید تنها با گوش دادن به
بخش کوتاهی از یک قطعه توانایی تشخیص خواننده را دارید. در حال حاضر فروشگاه‌های
موسیقی نیاز به کمک‌گیری از متخصصان دارند تا این اطلاعات را برای آن‌ها مشخص کنند
ولی این برای میلیون‌ها قطعه کاری طاقت فرسا و گاها غیرقابل اعتماد است. از
کاربردهای چنین سیستمی می‌توان به توصیه آهنگ‌های جدید، دسته‌بندی قطعات و حفاظت
از حقوق ناشر اشاره کرد.

\subsection{دسته‌بندی ژانر}
ژانر یکی دیگر از مفاهیم مهم برای دسته‌بندی قطعات موسیقی است و حجم عظیمی از
فروشگاه‌های موسیقی قطعات خود را بر اساس ژانر دسته‌بندی می‌کنند. این دسته‌بندی
معمولا بر اساس الگوهای موسیقیایی و سازهای استفاده شده انجام می‌شود. در حالی که
این تقسیم‌بندی برای شنوندگان عادی مسئولیت ساده‌ای نیست متخصصان بر اساس تجربه و
مهارت این مهم را انجام می‌دهند
\cite{perrot1999scanning,scaringella2006automatic}%
.

\subsection{جست‌وجو با زمزمه}
عکس و صوت پراستفاده‌ترین نوع محتوای چندرسانه‌ای هستند. در حاضر موتورهای
جست‌وجوی متنی به صورت گسترده استفاده می‌شوند و به صورت مستقیم در دسترس کاربران
هستند. با این حال زمانی که جست‌وجو بر روی داده‌های چندرسانه‌ای باشد، تبدیل
جستار چندرسانه‌ای به جستاری متنی فرآیندی پیچیده است و همچنین برای مواردی
مانند پیدا کردن یک قطعه موسیقی از طریق زمزمه موزیک آن شدنی نیست. به کمک یک سیستم
CB-MIR
می‌توان جست‌وجوهایی از این دست را ممکن کرد
\cite{yoshitaka1999survey}%
.

\subsection{تشخیص بار احساسی}
تشخیص بار احساسی موسیقی از روی الگوهای موسیقیایی یکی دیگر از مسائل هست که
کاربردهایی مانند پیشنهاد آهنگ بر اساس وضع احساسی کاربر دارد. هدف نهایی این است
که قطعه‌های موسیقی بر احساس بار احساسی به دسته‌های مانند شاد، غمگین، عصبی و
غیره تقسیم بندی شوند. به تازگی
MIREX
دیتاستی برای این مسئله منتشر کرده است که از آن به عنوان معیاری برای سنجش کارایی
سیستم‌ها استفاده شود، ولی حتی این دیتاست هم بسیار محدود است و قابل تعمیم به همه
سبک‌های موسیقی نیست. از این جهت نیاز به دیتاستی مرجع و جامع در این مسئله حس
می‌شود. همچنین تشخیص بار احساسی یک قطعه به واسطه جنبه‌های روانی مختلف موسیقی،
مسئله‌ای مبهم است.

\subsection{تشخیص ساز موسیقی}
تشخیص سازهای استفاده شده در یک قطعه، برای مسائل دیگر مانند تشخیص سبک یا بار
احساسی، حیاتی است. به صورت کلی هر قسمت یک قطعه یا تک صدا است که در آن فقط یک
ساز نواخته می‌شود یا چند صدا هست که در آن چندین ساز نواخته می‌شوند. مسئله
تشخیص ساز موسیقی یک مسئله دسته بندی چند برچسبی بر روی یک رشته است زیرا سیستم
باید با شنیدن یک قطعه موسیقی تشخیص دهد که چه سازهایی در هر قسمت نواخته می‌شود
و در صورت چند صدایی بودن بخش همزمان می‌تواند چندین ساز را انتخاب کند. سختی این
مسئله به علت گستردگی و شباهت دایره زنگی سازهای مختلف هست. همچنین چند صدایی یک
بخش سختی تشخیص را چند برابر می‌کند. از همین جهت اکثر تحقیقات انجام شده در این
حوزه بر روی قطعه‌های تک صدا انجام شده است.

\subsection{آوانویسی یک قطعه}
آوانویسی یک قطعه یکی دیگر مسائل این حوزه است. با توجه به ماهیت فیزیکی صوت و
فیزیک سازهای موسیقی، این مسئله جزو سخت‌ترین مسائل شناخته می‌شود. در ادامه به
صورت مفصل در مورد این مسئله بحث می‌شود و چند مورد از کارهای انجام شده در این
مسئله بررسی می‌شود.


\section{آوانویسی خودکار موسیقی}
آوانویسی خودکار موسیقی
\LTRfootnote{Automatic music transcription}
یا به اختصار
AMT
یکی از مسائل بنیادین حوزه بازیابی اطلاعات موسیقی است. هدف این مسئله تولید
خروجی نمادین و نت مانند از روی سیگنال صوتی یک قطعه چند صدا است. آوانویسی
موسیقی حتی برای متخصصان موسیقی هم مسئله سختی محسوب می‌شود و دقت سیستم‌های
کنونی از دقت انسانی کمتر است 
\cite{klapuri2007signal}.
آوانویسی چند صدایی به این علت مسئله سختی است که ترکیب چند نت که در یک لحظه
نواخته می‌شوند، باعث ترکیب و هم‌رخداد هارمونی‌ها در سیگنال صوتی می‌شوند.
همچنین تنوع سازهای نواخته شده نیز باعث تنوع در سیگنال ورودی می‌شود. همچنین
در صورتی که سیستم محدودیتی بر روی چندصدایی اعمال نکند، فضای خروجی ترکیبی آن‌ها
بسیار گسترده می‌شود که باعث پیچیدگی بیشتر مسئله می‌گردد. به صورت کلی تنوع در
سیگنال ورودی به مدل‌هایی داده می‌شود که هدفشان یادگیری خصوصیات دایره زنگی‌ای
است که آوانویسی می‌کنند
\cite{berg2014unsupervised,benetos2012shift}
در حالی که مشکل فضای خروجی زیاد با محدود کردن تعداد نت‌ها کنترل می‌شود
\cite{klapuri2003multiple,emiya2008automatic}.
آوانویسی چند صدایی پیانو علاوه بر سختی آوانویسی چند صدایی عادی، به علت شکلی که
انرژی صوت حاصل از فشردن هر کلاویه، تغییر می‌کند حتی مسئله‌ای سخت‌تر است. به
همین علت مدل نهایی باید بتواند خود را با سیگنال‌های با قدرت و هارمونیک مختلف
وقف دهد.

بسیاری از مسائل حوزه بازیابی اطلاعات موسیقی در صورتی که به جای سیگنال صوتی،
حالت نمادین موسیقی را به عنوان ورودی بگیرند، ساده‌تر می‌توانند حل شوند. به
عنوان مثال تشخیص بار احساسی آهنگ در صورتی که سیستم ورودی حالت نمادین موسیقی
را بگیرد، خیلی ساده‌تر می‌تواند الگوهای موسیقی را تشخیص دهد در نتیجه سیستم در
نهایت دقت بالاتری خواهد داشت. همچنین گستردگی بیشتر موسیقی‌هایی که حالت نمادین
آن‌ها موجود است، می‌تواند باعث گستردگی بیشتر مطالعات موسیقی شناسی محاسباتی%
\LTRfootnote{Computational musicology}
شود
\cite{cuthbert2010music21}.

فاکتورگیری نامنفی ماتریس یک روش پرطرفدار در روزهای ابتدایی مسئله آوانویسی بود
\cite{smaragdis2003non}،
ولی با پیشرفت‌های اخیر در حوزه یادگیری ژرف،‌ شبکه‌های عصبی مصنوعی توجه زیادی
را در حوزه بازیابی اطلاعات موسیقی جلب کرده‌اند. به صورت دقیق‌تر
شبکه‌های عصبی پیچشی به علت موفقیتشان در مسئله‌ی دسته بندی تصویر باعث استفاده
گسترده‌شان در این حوزه شده‌اند. به این علت که نمایش دو بعدی زمان-فرکانس یک
نمایش رایج برای صوت است. همچنین با توجه به استفاده گسترده از ساختار شبکه
عصبی بازگشتی در حوزه بازشناسی گفتار، از این معماری نیز برای آوانویسی استفاده
می‌شود. به این صورت که شبکه‌های پیچشی وظیفه مدلسازی آکوستیک و شبکه‌های بازگشتی
برای مدلسازی موسیقیایی استفاده می‌شود.

نشان داده شده است
\cite{kelz2016potential}
که بهره‌گیری از پیشرفت‌های اخیر در روش‌های یادگیری و ریگولاریزیشن و همچنین
تنظیم درست ابر مقدارها می‌تواند تاثیر به سزایی در دقت مدل داشته باشد. همچنین
توسط
\cite{sigtia2016end}
این ایده مطرح شده است که همچون بازشناسی گفتار، علاوه بر تنها یک مدل آکوستیک،
یک مدل زبانی هم آموزش داده شود که مدل زبانی یا در این جا مدل موسیقی احتمال
وقوع یک دنباله از نت‌ها را نشان می‌دهد. در  نهایت آوانویسی نهایی حاصل هر مدل
است به این صورت که مدل موسیقی احتمال انتخاب شدن دنباله‌ای از نت‌ها که معمولا
با هم ظاهر نمی‌شوند را کاهش می‌دهد و باعث می‌شود دقت نهایی افزایش یابد.

برای رفع مشکل ماندن صدا در ساز پس از فشردن کلاویه و هارمونی ایجاد شده از بقیه
نت‌ها، پیشنهاد شد
\cite{hawthorne2017onsets}
که مدل دو خروجی داشته باشد. خروجی اول وظیفه تشخیص زمان فشرده شدن کلاویه را
دارد و خروجی دوم، با شرطی شدن بر روی خروجی اول، وظیفه تشخیص نت را دارد. به این
شکل که در صورتی که خروجی اول فشرده شدن کلاویه را تشخیص ندهد، نتی از خروجی دوم
ثبت نمی‌شود. همچنین پیشنهاد شد که زمان شروع و پایان هر نت هر دو اهمیت دارند،
بر خلاف کارهای قبلی که تنها بر روی یکی از دو مورد دقت می‌شد. دلیل این تصمیم
این است که در درک ما از موسیقی هم زمان شروع یک نت و هم کشیدگی آن دارای اهمیت
است.

\subsection{دیتاست‌ها}
\subsection{MAPS}
برای این مسئله، MAPS
\LTRfootnote{MIDI Aligned Piano Sounds}
حکم دیتاست مرجع را دارد
\cite{emiya2010multipitch}.
این دیتاست شامل حدود ۳۱ گیگابایت فایل صوتی ضبط شده‌ است. برای ضبط از یک
پیانو مجازی و یک پیانو
\lr{Yamaha Disklavier}
استفاده شده است که در ۹ شرایط مختلف این قطعات ضبط شده‌اند. همچنین حالت نمادین
برای تمام قطعات در فرمت
MIDI
و فایل‌های متنی قرار داده شده است. صوت‌های ضبط شده متعلق به یکی از دسته‌های
زیر هستند:
\begin{itemize}
	\item نت‌های جدا شده و تک صدا
	\item آکوردهای تصادفی
	\item آکوردهای رایج
	\item قطعات موسیقی
\end{itemize}

این دیتاست تحت مجوز
\lr{Creative Common}
منتشر شده‌ است.

\subsection{MAESTRO}
MAESTRO%
\LTRfootnote{MIDI and Audio Edited for Synchronous TRacks and Organization}
دیتاست دیگری هست که توسط
\cite{hawthorne2018enabling}
معرفی شد. این دیتاست شامل بیش از ۱۷۲ ساعت اجرای پیانو هست که با دقت خیلی بالایی،
حدود ۳ میلی‌ثانیه، آوانویسی شده است. فایل‌های
MIDI
شامل اطلاعت شدت فشردن کلاویه‌ها و تغییرات پدال نگه‌دارنده هستند. همچنین هر قطعه با اطلاعاتی
همچون نام آهنگساز و سال اجرا حاشیه نویسی شده است.

برای این دیتاست همچنین ناحیه‌های آموزش، اعتبارسنجی و آزمایش مشخص شده‌اند به
گونه‌ای که قطعه‌ای یکسان حتی اگه اجراهای مختلفی از آن موجود باشد، در ناحیه‌هایی
متفاوت ظاهر نشود. اطلاعات هر ناحیه در جدول زیر نشان داده شده است.

\begin{center}
    \begin{tabular}{| c | c | c | c | c |}
        \hline
        قسمت & تعداد اجرا & تعداد قطعه & مدت به ساعت & حجم به گیگابایت
        \\ \hline
        آموزش & 954 & 295 & 140.1 & 83.6
        \\ \hline
        اعتبار سنجی & 105 & 60 & 15.3 & 9.1
        \\ \hline
        آزمایش & 125 & 75 & 16.9 & 10.1
        \\ \hline
        کل & 1184 & 430 & 172.3 & 102.8
        \\ \hline
    \end{tabular}
\end{center}

این دیتاست تحت مجوز
\lr{CC BY-NC-SA 4.0}
منتشر شده‌ است.