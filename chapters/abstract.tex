\chapter*{چکیده}
در بازشناسی خودکار گفتار استفاده از یک مدل زبانی در کنار مدل آوایی می‌تواند
عملکرد نهایی سیستم را به شدت افزایش دهد. هر چند موسیقی نیز مانند زبان طبیعی،
ساختاری کاملا قائده‌مند است، به علت محدودیت‌ها و چالش‌ها آوانویسی خودکار موسیقی،
مانند دنباله‌های ورودی و خروجی به شدت طولانی‌تر، استفاده از شیوه رایج ارتباط دو
مدل زبانی و آوایی در این مسئله ممکن نمی‌باشد. در این پایان‌نامه ابتدا یک مدل
موسیقیایی معرفی شده است. سپس یک شیوه ارتباطی جدید برای اتصال مدل موسیقیایی با
مدل آوایی بررسی شده است. در نهایت عملکرد نهایی سیستم، جهت اطمینان از کارایی، با
یک مدل آوایی تنها مقایسه شده است.