\chapter*{چکیده}
«آوانویسی موسیقی» به فرایند گفته می‌شود که در آن تلاش می‌شود حالت شنیداری
موسیقی تبدیل به حالتی نمادین و نوشتاری شود. این حالت نوشتاری می‌توانند به
موسیقیدانان کمک کند که درکی بهتری از موسیقی پیدا کنند. همچنین آموختن نواخت آن
قطعه را برای نوازنده بسیار تسهیل می‌بخشد.

به دلیل پیچیدگی ذاتی، انجام این فرایند نیازمند درکی عمقی از موسیقی است و حتی
برای موسیقی‌دانان با تجربه نیز کاری زمان بر و خطا خیز محسوب می‌شود. «آوانویسی
خودکار موسیقی» تلاش می‌کند این فرایند پیچیده را بدون دخالت انسان، و با بهره بری
از روش‌های پردازش سیگنال و یادگیری ماشینی، به انجام برسناد.

در این پایان‌نامه تلاش می‌شود یک مدلی مبتی بر یادگیری عمقی برای حل این مسئله
بررسی شود.