\chapter{پیشنهادات و نتیجه‌گیری}
\section{پیشنهادات}
بزرگ‌ترین مانع بر سر راه آزمایش کامل ساختار معرفی شده محدودیت‌های زیرساخت
محاسباتی بوده است. به علت این محدودیت‌ها یک آموزش کامل متاسفانه ممکن نبوده است و
پروسه آموزش قبل از همگرایی کامل مدل متوقف شده است. مقایسه برابر با سایر مدل‌های
پیشنهاد شده نیاز به یک زیرساخت محاسباتی قدرت‌مند دارد تا مدل بتواند به صورت کامل
آموزش داده شود. همچنین با توجه به این که مدل هیچ نشانه‌ای از \gls{overfit} از
خود نشان نداده است می‌توان مدل را بزرگ‌تر کرد تا الگوهای پیچیده‌تر را نیز فرا
بگیرد که این امر نیز مجددا نیازمند زیرساخت محاسباتی مناسب‌تری است.

یکی از اصلی‌ترین مزایای استفاده از یک مدل موسیقیایی توانایی این مدل در آموزش
دیدن بر روی داده‌های آوانویسی تنها است. در این پایان‌نامه برای آموزش مدل
موسیقیایی تنها از اطلاعات آوانویسی موجود در \gls{dataset} MAESTRO استفاده شده
است. با این وجود \glspl{dataset} بسیار بزرگتری برای آوانویسی تنها موجود
می‌باشند. متاسفانه امکان استفاده از این \gls{dataset} نیز به علت محدودیت‌های
زیرساخت محاسباتی ممکن نبوده است. ولی قطعا یک آموزش جامع‌تر می‌تواند عملکرد مدل
را به شکل محسوسی بهبود ببخشید.

با توجه به ماهیت مسئله، دنباله اطلاعات ورودی و خروجی مدل به صورت ذاتی بسیار
طولانی‌تر از آن‌هستند که بتوان آن‌ها را به صورت کامل به مدل داد. برای حل این
محدودیت هر دنباله به دنباله‌های کوچیک‌تری با طول حداکثر هزار تقسیم شده است. هر
چند یک ورودی با هزار فرم دنباله‌ای بسیار طولانی ممکن است به نظر برسد ولی در عمل
به مدل تنها اطلاعات ۳۲ ثانیه از یک قطعه موسیقی را می‌تواند منتقل کند. با وجود
این که این طول برای درک روابط کوتاه مدت بین کلمات موسیقیایی نزدیک به هم کافی
هست، ولی برای مدل‌سازی ارتباطات طولانی‌تر در یک قطعه موسیقیایی اصلا مناسب
نیست. در موسیقی نه تنها کلمات کنار هم با هم رابطه دارند و از هم تبعیت
می‌کنند بلکه هر جمله بر روی کل جملات یک قطعه تاثیرگذار است و حتی میان اولین
نت‌های قطعه و آخرین نت‌های آن نیز روابطی وجود دارد.

اخیرا نسخه‌های جدیدی از \gls{transformer} ارائه شده است که می‌تواند جملات بسیار
طولانی‌تر را به عنوان ورودی دریافت کند. از معروف‌ترین این ساختارها می‌توان به
reformer \cite{kitaev2020reformer} اشاره کرد که می‌تواند دنباله‌هایی به طول تا
یک میلیون را نیز پردازش کند. هر چند این ساختارها به شدت جدید محسوب می‌شوند و
همچنان در دست بررسی هستند ولی قطعا یک مدل موسیقیایی که بتواند کل یک قطعه را به
صورت همزمان ببیند و روابط به هر طول را فراگیرد می‌تواند نتایج بسیار بالاتری
ارائه دهد.

یکی از مزایای بزرگ یک مدل موسیقایی این است که هیچ وابستگی به یک ساز خاص ندارد و
یک مدل موسیقیایی که به صورت جمع آموزش داده شود می‌تواند بر روی هر ساز موسیقی یا
حتی مجموع چندین ساز با هم اعمال شود. متاسفانه در حال حاضر تقریبا تمام
\glspl{dataset} موجود بر روی پیانو ضبط شده‌اند. ولی در صورت وجود \gls{dataset}
برای سازهای دیگر، تنها کافی است تا بخش اولیه مدل برای سازهای مختلف آموزش ببیند
تا بتواند عملایت آوانویسی را برای سازهای دیگر نیز انجام دهد.

\section{نتیجه‌گیری}
یکی از مشکلات اصلی موجود در حوزه \gls{atm} همچنان محدودیت \gls{dataset} موجود
هست. با وجود این که \gls{dataset} MAESTRO شامل حجم زیادی اطلاعات است ولی
مدل‌های آموزش دیده با این \gls{dataset} تنها در آوانویسی قطعات موسیقی کلاسیک
اجرا شده بر روی پیانو توانایی قابل قبول خواهند داشت. همچنین با توجه به پیچیدگی و
هزینه‌بر بودن فرایند جمع‌آوری \gls{dataset} مناسب برای مسئله احتمالا تلاش برای
ایجاد \gls{dataset} کامل‌تری زیاد منطقی نباشد.

یک مدل موسیقیایی می‌تواند با استفاده از آوانویسی تنها آموزش ببین تا قوانین موجود
در قطعات موسیقی را فرابگیرد. از طرفی دیگر داده آوانویسی تنها به شدت رایج‌تر است
و با هزینه‌ به شدت کمتر می‌توان \gls{dataset} کامل برای آموزش یک مدل موسیقیایی
فراهم شود. همچنین قوانین موسیقی از ساز کاملا مستقل هستند و یک مدل موسیقیایی
می‌تواند به شکل کاملا یکسانی برای سازهای مختلف موثر باشد.

هرچند استفاده از یک مدل موسیقیایی که با استفاده از قوانین موسیقیایی بتواند خروجی
مدل آوایی را تصحیح کند قطعا در افزایش دقت و کارایی سیستم‌های \gls{atm} تاثیر
بسیاری خواهد داشت، ولی به علت چالش‌هایی مانند طولانی بودن طول دنباله‌های ورودی
استفاده از روش‌ها رایج در حوزه \gls{asr} برای ارتباط بین دو مدل موسیقیایی و
آوایی ممکن نیست.

روش پیشنهاد شده در این مقاله یک رویکرد متفاوت برای اتصال این دو مدل به هم است.
هرچند که یک آموزش کامل به علت محدودیت‌های زیرساختی محاسباتی ممکن نبود، ولی نتایج
به دست آمده به خوبی گواه بر پتانسیل بالا این شیوه اتصال یک مدل موسیقیایی به مدل
آوایی است.